\documentclass{article}[10pt]

%%----PACKS----%%
% misc
\usepackage[margin=0.5in]{geometry}
% math
\usepackage{amsmath,
            amsfonts,
            amssymb}
% images
\usepackage{graphicx}
% code
\usepackage{listings,
            minted}
\usepackage{inconsolata}
%%-------------%%

%%--SETTINGS---%% 
% minted
\usepackage{xcolor}
\definecolor{cppbg}{rgb}{10,240,240}
\newminted[cpp]{C++}{linenos=true, texcl=true, bgcolor=cppbg, tabsize=4 , frame=lines}
\newminted[bashing]{zsh}

%%--CUSTOM-----%%
\newenvironment{question}
{
\noindent \bf Problem statement. \rm
}


\newenvironment{solution}
{
\noindent \bf Solution. \rm
}

\begin{document}

\begin{flushright}
	\textbf{Regular: Jaap van der Leest\\  Attendee: Timon van der Berg \\ }
\today
\end{flushright}

\begin{center}
\textbf{C++ Course \\
Assignment 3} \\
\end{center}
\section*{Exercise 18}

- It is appropriate to use an int-type parameter when only whole numbers are used. 
In the code example this is illustrated. The price of pizzas is calculated based on the number of pizzas ordered.
\begin{cpp}
    int nrPizzas = 7;
    cout << "Give the pizzadeliverguy: " << nrPizzas * 4.99 << " euros.\n";
\end{cpp}  
  
- It is appropriate to use an std::string value parameter when pieces of text need to be stored. 
In the code example this is illustrated because some text is predefined and later shown.
\begin{cpp}
    std::string welcome = "Hello to all of you!";
    cout << welcome;
\end{cpp}

- It is appropriate to use a const reference to an int-type parameter... Well it is not really appropriate to define a const reference. Because a reference to a variable is always a const and the compiler won't accept it either.

- It is appropriate to use a const reference to a std::string value parameter... Well it is not really appropriate to define a const reference. Because a reference to a variable is always a const and the compiler won't accept it either.


- It is appropriate to use a non-const reference to an int-type parameter if we want to make a reference to an int-type parameter. A reference is always constant so no need to declare const or non const.
A reference to an int-type parameter would make sense if we want to pass the value of the int-type parameter to a function but not copy it. This can be for reasons of e.g. memory or speed, or if we want to access the int-type parameter (which lives somewhere else locally) from within the function.

\begin{cpp}
void ShowNumber(int &number)
{
    cout << number << '\n';
}

int main()
{
    int value = 5;
    ShowNumber(value);
}
\end{cpp}

- It is appropriate to use a non-const reference to a std::string value parameter if we want to make a reference to a std::string value parameter. A reference is always constant so no need to declare const or non const.
A reference to a std::string value parameter would make sense if we want to pass the value of the std::string value parameter to a function but not copy it. This can be for reasons of e.g. memory or speed, or if we want to access the std::string value parameter (which lives somewhere else locally) from within the function.

\begin{cpp}
void ShowText(std::string &text)
{
    cout << text << '\n';
}

int main()
{
    std::string string = "Hello world!";
    ShowText(string);
}
\end{cpp}

- It is appropriate to use a const rvalue reference to a int type parameter if... no. A const rvalue is nonsense. a rvalue by definition is temporary and will cease to exist after it is used. 


- It is appropriate to use a const rvalue reference to a std::string parameter if... no. A const rvalue is nonsense. a rvalue by definition is temporary and will cease to exist after it is used. 


- It is appropriate to use a rvalue reference to a int type parameter if we need to use (and probably modify) an int only within a function which value is passed to it when calling the function. It ceases to exist after the function ends.
\begin{cpp}
void myFun(int &&number)
\end{cpp}

- It is appropriate to use a rvalue reference to a std::string parameter if we need to use (and probably modify) a string only within a function which value is passed to it when calling the function. It ceases to exist after the function ends.

\begin{cpp}
void myFun(string &&myString)
\end{cpp}

- It is appropriate to return an int-type value if a function returns a whole number.
\begin{cpp}
int multiply(int first, int second)
{
    return (first * second);
}
\end{cpp}

- It is appropriate to return a std::string value if a function returns a piece of text.

\begin{cpp}
std::string helloWorld(void)
{
    std::string hello = "Hello World!\n";
    return(hello);
}
\end{cpp}

- It is not appropriate to return something like a reference or rvalue reference (const or non-const)
Because the values being returned are not accessible anymore when the function ends. 



\section*{Exercise 19}
makefile:
\inputminted[linenos=true, bgcolor=cppbg, tabsize=4 , frame=lines]{make}{../19/makefile}
myheader.ih:
\inputminted[linenos=true, bgcolor=cppbg, tabsize=4 , frame=lines]{cpp}{../19/myheader.ih}
exercise\_19.cc:
\inputminted[linenos=true, bgcolor=cppbg, tabsize=4 , frame=lines]{cpp}{../19/exercise_19.cc}
method1.cc:
\inputminted[linenos=true, bgcolor=cppbg, tabsize=4 , frame=lines]{cpp}{../19/method1.cc}
method2.cc:
\inputminted[linenos=true, bgcolor=cppbg, tabsize=4 , frame=lines]{cpp}{../19/method2.cc}
method3.cc:
\inputminted[linenos=true, bgcolor=cppbg, tabsize=4 , frame=lines]{cpp}{../19/method3.cc}
method4.cc:
\inputminted[linenos=true, bgcolor=cppbg, tabsize=4 , frame=lines]{cpp}{../19/method4.cc}
method5.cc:
\inputminted[linenos=true, bgcolor=cppbg, tabsize=4 , frame=lines]{cpp}{../19/method5.cc}
method6.cc:
\inputminted[linenos=true, bgcolor=cppbg, tabsize=4 , frame=lines]{cpp}{../19/method6.cc}
\section*{Exercise 20}
 
head.ih:
\inputminted[linenos=true, bgcolor=cppbg, tabsize=4 , frame=lines]{cpp}{../20/head.ih}
main.cc:
\inputminted[linenos=true, bgcolor=cppbg, tabsize=4 , frame=lines]{cpp}{../20/main.cc}
arguments.cc:
\inputminted[linenos=true, bgcolor=cppbg, tabsize=4 , frame=lines]{cpp}{../20/arguments.cc}
process.cc:
\inputminted[linenos=true, bgcolor=cppbg, tabsize=4 , frame=lines]{cpp}{../20/process.cc}
selectopt.cc:
\inputminted[linenos=true, bgcolor=cppbg, tabsize=4 , frame=lines]{cpp}{../20/selectopt.cc}
usage.cc:
\inputminted[linenos=true, bgcolor=cppbg, tabsize=4 , frame=lines]{cpp}{../20/usage.cc}
version.cc:
\inputminted[linenos=true, bgcolor=cppbg, tabsize=4 , frame=lines]{cpp}{../20/version.cc}

 

\section*{Exercise 21}
 
 

 
myheader.ih:
\inputminted[linenos=true, bgcolor=cppbg, tabsize=4 , frame=lines]{cpp}{../21/myheader.ih}
main.cc:
\inputminted[linenos=true, bgcolor=cppbg, tabsize=4 , frame=lines]{cpp}{../21/main.cc}
partial.cc:
\inputminted[linenos=true, bgcolor=cppbg, tabsize=4 , frame=lines]{cpp}{../21/partial.cc}
square.cc
\inputminted[linenos=true, bgcolor=cppbg, tabsize=4 , frame=lines]{cpp}{../21/square.cc}
 

\section*{Exercise 22}
 
 

 
test.sh (a small shell script to test the program)
\inputminted[linenos=true, bgcolor=cppbg, tabsize=4 , frame=lines]{bash}{../22/test.sh}
header.ih:
\inputminted[linenos=true, bgcolor=cppbg, tabsize=4 , frame=lines]{cpp}{../22/header.ih}
main.cc
\inputminted[linenos=true, bgcolor=cppbg, tabsize=4 , frame=lines]{cpp}{../22/main.cc}
decode.cc
\inputminted[linenos=true, bgcolor=cppbg, tabsize=4 , frame=lines]{cpp}{../22/decode.cc}
dectohex.cc
\inputminted[linenos=true, bgcolor=cppbg, tabsize=4 , frame=lines]{cpp}{../22/dectohex.cc}
encode.cc:
\inputminted[linenos=true, bgcolor=cppbg, tabsize=4 , frame=lines]{cpp}{../22/encode.cc}
getopt.cc:
\inputminted[linenos=true, bgcolor=cppbg, tabsize=4 , frame=lines]{cpp}{../22/getopt.cc}
hextodec.cc
\inputminted[linenos=true, bgcolor=cppbg, tabsize=4 , frame=lines]{cpp}{../22/hextodec.cc}
isalpha.cc
\inputminted[linenos=true, bgcolor=cppbg, tabsize=4 , frame=lines]{cpp}{../22/isalpha.cc}
isother.cc
\inputminted[linenos=true, bgcolor=cppbg, tabsize=4 , frame=lines]{cpp}{../22/isother.cc}
usage.cc
\inputminted[linenos=true, bgcolor=cppbg, tabsize=4 , frame=lines]{cpp}{../22/usage.cc}
 

\section*{Exercise 23}

 
 

 
header.ih:
\inputminted[linenos=true, bgcolor=cppbg, tabsize=4 , frame=lines]{cpp}{../23/header.ih}
main.cc:
\inputminted[linenos=true, bgcolor=cppbg, tabsize=4 , frame=lines]{cpp}{../23/main.cc}
printbig.cc:
\inputminted[linenos=true, bgcolor=cppbg, tabsize=4 , frame=lines]{cpp}{../23/printbig.cc}
printbigdirect.cc
\inputminted[linenos=true, bgcolor=cppbg, tabsize=4 , frame=lines]{cpp}{../23/printbigdirect.cc}
usage.cc:
\inputminted[linenos=true, bgcolor=cppbg, tabsize=4 , frame=lines]{cpp}{../23/usage.cc}

 
\end{document}

