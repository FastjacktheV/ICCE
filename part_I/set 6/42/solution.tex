% latex to be imported

\section*{Exercise 42}

\begin{question}
What are the variants of new/delete? For each of the variants provide a (short!) example in which the used new/delete is appropriate and provide a short explanation why it is appropriately used. 

\end{question}
\begin{solution}
\subsection*{new/delete}
\texttt{new} is used to allocate memory for a primitive type or object. When allocating for an object it will call the object constructor. Example of \texttt{new} use:
    \begin{minted}{cpp}
    int *ptr = new int;
    \end{minted}
This allocation is appropriate since we want to allocate memory for a single int.

\texttt{delete} is used to deallocate memory that was allocated using \texttt{new}. 
If called on an object (not a primitive type) it will also call that objects' destructor.
Example of \texttt{delete} use:
    \begin{minted}{cpp}
    std::string *ptr = new std::string; 
    delete ptr;
    \end{minted}
This is appropriate because \texttt{delete} is used on memory allocated using \texttt{new}.
\end{solution}

\subsection*{new[], delete[]}
\texttt{new[]} is used to allocate memory for arrays. Like \texttt{new} it is type-safe: the type of the element has to be declared. Like \texttt{new}, it calls constructors. An example of using \texttt{new}:
    \begin{minted}{cpp}
    int *aoi = new int[20];
    \end{minted}
\texttt{delete[]} is used to delete memory allocated using \texttt{new[]}. Unlike \texttt{new}, \text{new[]} saves the size of the array it allocates. \texttt{delete[]} uses this to delete the array. Destructors are called\footnote{If the array contains a primitive type no destructors are called. Therefore an array of pointers require manual destruction of whatever is pointed to.}. An example of \texttt{delete[]} usage:
    \begin{minted}{cpp}
    string *strp = new string[550];
    delete[] strp;
    \end{minted}
This is appropriate because \texttt{delete[]} is used on an array allocated using \texttt{new[]}.

\subsection*{operator new, operator delete}
\texttt{operator new} is used to allocate raw bytes of memory. To actually use this memory, a static cast is required. An example of using \texttt{operator new}:
    \begin{minted}{cpp}
    size_t *sp = static_cast<size_t *>(operator new(5 * sizeof(size_t)));
    \end{minted}
    This is appropriate because \texttt{operator new} is used to allocate raw memory.
    Here we first calculate the number of bytes needed for 5 \texttt{size\_t} variables. We then allocate the memory. \texttt{operator new} does not care for types.
        
    \texttt{operator delete} is used to deallocate memory that was allocated using \texttt{operator new}. Like \texttt{operator new}, \texttt{operator delete} does not care for types. Because \texttt{operator new} saves the number of bytes allocated, \texttt{operator delete} knows how much memory to deallocate. \texttt{operator delete} does not call any destructors. An example of using \texttt{operator delete}:
    \begin{minted}{cpp}
    string *sp = static_cast<string *>(operator new(5 * sizeof(string)));
    operator delete(sp);
    \end{minted}
   This is appropriate because we are using operator delete to deallocate memory that was allocated using operator new.

\subsection*{placement new}
placement \texttt{new} is found in \texttt{<memory>} and overloads \texttt{new}.  Placement \texttt{new} is used to place objects in previously allocated memory.
An example of using placement \texttt{new}:
    \begin{minted}{cpp}
     string *sp = static_cast<string *>( operator new(15 * sizeof(string)));
     new sp string("Donald Knuth");
    \end{minted}
This is appropriate because we are using operator new on memory of the correct type that was previously allocated. 
We have placed a single string in this memory, leaving room for 14 more.
